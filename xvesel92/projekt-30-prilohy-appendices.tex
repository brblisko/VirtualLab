% Tento soubor nahraďte vlastním souborem s přílohami (nadpisy níže jsou pouze pro příklad)

% Pro kompilaci po částech (viz projekt.tex), nutno odkomentovat a upravit
%\documentclass[../projekt.tex]{subfiles}
%\begin{document}

% Umístění obsahu paměťového média do příloh je vhodné konzultovat s vedoucím
\chapter{Obsah priloženého pamäťového média}
Príloha obsahuje dátovú štruktúru priloženého média a popis jednotlivých priečinkov a súborov.

\begin{itemize}
    \item \verb|xvesel92.pdf| - Verzia bakalárskej práce vo formáte PDF.
    \item \verb|xvesel92/| - Priečinok obsahujúci zdrojové kódy pre textovú časť bakalárskej práce.
    \item \verb|daemon/| - Priečinok obsahujúci zdrojové kódy démona.
    \item \verb|Scripts/| - Priečinok obsahujúci skripty, ktoré sa používajú na zariadeniach PYNQ.
    \item \verb|web/| - Priečinok obsahujúci zdrojové kódy webovej aplikácie.
    \item \verb|VLab.sql| - Zdrojový kód generujúci databázu.
    \item \verb|README.md| - Informácie o aplikácii, inštalácie a spustení.
\end{itemize}



%\chapter{Manuál}

%\chapter{Konfigurační soubor}

%\chapter{RelaxNG Schéma konfiguračního souboru}

%\chapter{Plakát}
